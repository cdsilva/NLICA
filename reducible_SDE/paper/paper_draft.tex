\documentclass[1p]{elsarticle}

\usepackage{amsmath}
\usepackage{amsfonts}

\title{Data-driven Reduction of Multiscale Stochastic Dynamical Systems }

\author[PrincetonCBE]{C.J.~Dsilva}
\ead{cdsilva@princeton.edu}

\author[YaleMath]{R.~Talmon}
\ead{ronen.talmon@yale.edu}

\author[YaleMath]{R.R.~Coifman}
\ead{coifman@math.yale.edu}

\author[PrincetonCBE, PrincetonPACM]{I.G.~Kevrekidis\corref{cor1}}
\ead{yannis@princeton.edu}

\address[PrincetonCBE]{Department of Chemical and Biological Engineering, Princton University, Princeton, New Jersey, 08544, USA}
\address[YaleMath]{Department of Mathematics, Yale University, New Haven, Connecticut, 06520, USA}
\address[PrincetonPACM]{Program in Applied and Computational Mathematics, Princton University, Princeton, New Jersey, 08544, USA}

\cortext[cor1]{Corresponding author}


\begin{document}

\begin{abstract}

\begin{itemize}

\item Link between data mining and dynamical systems

\item Extracting or reducing an SDE system to its slow variables using the Mahalanobis graph and NIVs

\item Full mathematical analysis of the slow variable reduction procedure, including necessary conditions for accurate recovery of the slow variables and a framework for empirically setting the algorithm parameters

\end{itemize}

\end{abstract}


\begin{keyword}
 
\end{keyword}

\maketitle

\section{Introduction}

\section{Theory}

\subsection{Multiscale SDEs}

Multiscale SDE setting: fast/slow, want to recover slow variable(s) even when the fast variables are large/very noisy

\begin{equation}
\begin{aligned}
dx &= adt + dW_1\\
dy &= -\frac{y}{\epsilon} dt + \frac{1}{\sqrt{\epsilon}} dW_2
\end{aligned}
\end{equation}

So $y$ is a fast noise whose equilibrium measure is bounded and $\mathcal{O}(1)$

\subsection{Nonlinear Intrinsic Variables}

Background on NIVs 	

\begin{equation}
dx_i = a(X) dt + dW_i
\end{equation}
and we will let $X = \begin{bmatrix} X^1 \\ X^2 \\ \vdots \\ X^n \end{bmatrix}$

We want to emphasize the link between multi scale SDEs and NIVs: NIVs rescales the variables so that the ``fast'' variables become small

saturation time of SDE has to be smaller than the timescales



Let $y = \frac{z}{\sqrt{\epsilon}}$. Then
\begin{equation}
\begin{aligned}
dx &= adt + dW_1\\
dz &= -\frac{z}{\epsilon} dt +  dW_2
\end{aligned}
\end{equation}

Therefore, now $z$ is a stochastic variable whose diffusion is unity, and whose equilibrium measure is $\mathcal{O}(\epsilon)$. 
%
Therefore, when $\epsilon \ll 1$, we have that $z \ll x$, and so we will recover $x$ only.


\subsection{Errors and Estimation}

Error and estimation analysis of NIVs (remark that applies to more general settings): give conditions/bounds when we can recover the intrinsic slow variable

Empirical configuration/demonstration of the method, how to use the analysis to set the parameters in the algorithm


Given $X$ governed by the above SDE, we can sample ``bursts'' around a point $X_0$ in order to estimate the covariance. 

These clouds will be distributed 

\begin{equation}
X \sim \mathcal{N}\left( X_0, \delta t I \right)
\end{equation}


We now consider a function $\mathbf{f}: \mathbb{R}^n \mapsto \mathbb{R}^m$.
%
Let $f^i(X)$ denote the $i^{th}$ entry of $\mathbf{f}$, and let $f^i_j(X) = \frac{\partial f^i}{\partial X^j}$.
%
By Taylor expansion around $X=X_0$,
%
\begin{eqnarray}
\hat{C}(X_0) &=& \delta t J_1 J_1^T 
+ \frac{\delta t^2}{2} \left[ J_1 J_3^T + J_3 J_1^T  
+ \frac{3}{2} J_2 J_2^T 
-\frac{1 }{2} J_2 1 1^T J_2^T \right]
+ \mathcal{O} (\delta t^3) %\\
%&=& \delta t J_1 J_1^T 
%+ \frac{\delta t^2}{2} U \Sigma V
%+ \mathcal{O} (\delta t^3)
\end{eqnarray}
where
\begin{equation}
\begin{aligned}
J_1(X_0) &= \begin{bmatrix}
f_1^1 & f_2^1 & \dots & f_n^1 \\
\vdots & \vdots & \cdots & \vdots \\
f_1^m & f_2^m & \dots & f_n^m
\end{bmatrix}\\
%
J_2(X_0) &= \begin{bmatrix}
f_{11}^1 & f_{22}^1 & \dots & f_{nn}^1 \\
\vdots & \vdots & \cdots & \vdots \\
f_{11}^m & f_{22}^m & \dots & f_{nn}^m
\end{bmatrix}\\
%
J_3(X_0) &= \begin{bmatrix}
f_{111}^1 & f_{222}^1 & \dots & f_{nnn}^1 \\
\vdots & \vdots & \cdots & \vdots \\
f_{111}^m & f_{222}^m & \dots & f_{nnn}^m
\end{bmatrix}
\end{aligned}
\end{equation}
%
In this equation, we see what we expect (a linear function of the variance, due to the gradient of $f$) and a second (higher-order) term arising from the curvature and third derivative of $f$.
%
The importance of this second term increases as the curvature increases, as well as when $\delta t$ increases. 



Then
\begin{equation}
\hat{C}^{-1}(X_0) \approx \frac{1}{\delta t} \left( J J^T \right)^{-1} - A
\end{equation}
%
where 
\begin{equation}
A = \frac{1}{2} (J_1 J_1^T)^{-1} U \left(\Sigma^{-1} + \frac{\delta t}{2} V \left( J_1 J_1^T \right)^{-1} U \right)^{-1} V \left( J_1 J_1^T \right)^{-1}
\end{equation}
and
\begin{equation}
U \Sigma V = J_1 J_3^T + J_3 J_1^T + \frac{3}{2} J_2 J_2^T -\frac{1 }{2} J_2 1 1^T J_2^T 
\end{equation}
$A$ is a function of the first-, second-, and third-order derivatives of $\mathbf{f}$.


We now approximate the distances $\| X_2 - X_1 \|$.
%
Let $g = f^{-1}: \mathbb{R}^m \mapsto \mathbb{R}^n$, and let $Y_1 = f(X_1)$, and $Y_2 = f(X_2)$.
%
Again, we let $g_j^i = \frac{\partial g^i}{\partial Y^j}$.
%
Then, by Taylor expansion, we obtain
\begin{equation} \label{eq:taylor_expansion}
\begin{aligned}
&\| X_2 - X_1 \|^2 \\
\le&  \left| \frac{1}{2} (Y_2 - Y_1 )^T ((J J^T)^{-1} (Y_1) + (J J^T)^{-1}(Y_2)) (Y_2 - Y_1 ) \right| \\
& + n \left( \left| K_1 K_2 \right| + \left| \frac{ K_2^2}{4} \right|  + \left| \frac{K_1 K_3}{3} \right|  \right) \| Y_2 - Y_1 \| ^4  \\
& + \mathcal{O} (\|Y_1 - Y_2 \|^6 ) 
\end{aligned}
\end{equation}
%
where
%
\begin{equation}
\begin{aligned}
K_1 &= \sup_{i,j,Y} |g_j^i(Y)|\\
K_2 &= \sup_{i,j,k,Y} |g_{jk}^i(Y)|\\
K_3 &= \sup_{i,j,k,l,Y} |g_{jkl}^i(Y)|
\end{aligned}
\end{equation}

Substituting our estimate of the covariance into \eqref{eq:taylor_expansion}, we obtain
\begin{equation}
\begin{aligned}
&\| X_2 - X_1 \|^2 \\
\approx &  \frac{1}{2} (Y_2 - Y_1)^T \left((J J^T)^{-1}(Y_1)  + (J J^T)^{-1}(Y_2) \right) (Y_2 - Y_1) \\
& - \frac{\delta t}{2} (Y_2 - Y_1)^T \left( A(Y_1) + A(Y_2) \right) (Y_2 - Y_1)\\
& + n \left( \left| K_1 K_2 \right| + \left| \frac{ K_2^2}{4} \right|  + \left| \frac{K_1 K_3}{3} \right|  \right) \| Y_2 - Y_1 \| ^4  \\
& + \mathcal{O} (\|Y_1 - Y_2 \|^6 ) 
\end{aligned}
\end{equation}
%
We want the first term in this expression to dominate.


\section{Results}

 Toy problem results (half-moon): illustration of the analysis + extraction of slow variable and ``suppression'' of fast variable depending on $\delta t$ and saturation time of fast
 
 Define the following two-dimensional system of SDEs
\begin{eqnarray}
dX(1) &=& 3 dt + dW_1 \\ 
dX(2) &=& \frac{-X(2)}{\epsilon} dt + dW_2 
\end{eqnarray}
where $W_1, W_2$ are independent Brownian motions, and $\epsilon \ll 1$.

Let
\begin{eqnarray}
f^1(X) &=& X(1) + \frac{ X(2)^2}{\epsilon} \\
f^2(X) &=& \frac{X(2)}{\sqrt{\epsilon}}
\end{eqnarray}
Therefore, $X(2)$ is now a ``fast'' variable, and we have further applied a nonlinear transformation to this fast-slow system. 

The inverse function $g = f^{-1}$ is
\begin{eqnarray}
g^1(Y) &=& Y(1) - Y(2)^2 \\
g^2(Y) &=& Y(2) \sqrt{\epsilon}
\end{eqnarray}


 
\subsection{NIVs versus DMAPS}

DMAPS gives the fast, NIVs gives the slow 

\subsection{Determining Appropriate Parameters}

Illustration of different aspects (error terms) in the analysis

The three relevant quantities for NIVs are 
\begin{itemize}
\item The linearized approximation to the distance,
%
\begin{equation}
(Y_2 - Y_1)^T (JJ^T)^{-1} (Y_2 - Y_1) 
\approx \mathcal{O} \left(  \frac{1}{2} \left( 2 + \epsilon + \sqrt{ 9 + 2 \epsilon + \epsilon^2}\right) \Delta Y^2\right) 
\end{equation}
where $\Delta Y =  \|Y_2 - Y_1\|$

\item The error from the {\em estimation} of the covariance.
%
This is
\begin{equation}
\frac{\delta t}{2} (Y_2 - Y_1)^T A (Y_2 - Y_1) \approx \mathcal{O} \left( \frac{38 \delta t}{\epsilon ^2 + 2 \delta t}\Delta Y^2 \right)
\end{equation}

\item The errors from the {\em truncation} of the Taylor expansion for the Mahalanobis distance, which is 
\begin{equation}
n \left( \left| K_1 K_2 \right| + \left| \frac{ K_2^2}{4} \right|  + \left| \frac{K_1 K_3}{3} \right|  \right) \| Y_2 - Y_1 \| ^4 \approx \mathcal{O} \left( 10 \Delta Y^4  \right) 
\end{equation}

\end{itemize}

\subsection{Recovery of Fast Variable}

Show empirically what happens as we change the neighborhood (example from notes 1) (recovery of fast variable deep in the spectrum)

\section{Conclusion}

We showed that in certain cases (when we do {\em not} have a simulator where we can change $\delta t$), the data cannot be processed as-is (we cannot find the right kernel scale given a fixed $\delta t$ such that we can accurately recover the slow variable). 

If the cloud of samples is too big then we can observe the cloud of clouds (and those clouds can be histograms, Fourier, scattering, etc.) as a way to get smaller clouds. 


Richardson extrapolation allows us to get an estimate of a second-order term in the covariance estimation, thereby locally approximating the function using a quadratic form, rather than a linear form,  which can lead to a better/improved/more accurate ``Mahalanobis'' metric. 



\end{document}
