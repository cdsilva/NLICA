\documentclass[12pt]{article}

\usepackage{amsmath}
\usepackage{amsfonts}
\usepackage[margin=1in]{geometry}
\usepackage{graphicx}
\usepackage{epstopdf}
\graphicspath{ {quadratic_example/} }

\title{Data-driven Reduction of Multiscale Stochastic Dynamical Systems }

\begin{document}

\maketitle

\section{Outline}

Main message/contribution:
\begin{itemize}

\item We want to emphasize the link between multi scale SDEs and NIVs: NIVs rescales the variables so that the ``fast'' variables become small

\item 

\end{itemize}

\begin{itemize}

\item Introduction

\item Multiscale SDE setting 

\item Analysis of NIVs (remark that applies to more general settings)

\item Toy problem results (half-moon): illustration of the analysis + extraction of slow variable and ``suppression'' of fast variable depending on $\delta t$ and saturation time of fast

\item Conclusion: 

If the cloud of samples is too big then we can observe the cloud of clouds (and those clouds can be histograms, Fourier, scattering, etc.) as a way to get smaller clouds. 


Richardson extrapolation allows us to get an estimate of a second-order term in the covariance estimation, thereby locally approximating the function using a quadratic form, rather than a linear form,  which can lead to a better/improved/more accurate ``Mahalanobis'' metric. 


\end{itemize}



\end{document}
